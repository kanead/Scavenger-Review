\documentclass[a4paper,12pt]{article}
\usepackage{ecography}
\usepackage{lmodern}

 \renewcommand{\familydefault}{\sfdefault}


\title{The natural history of scavenging in vertebrates}
\running{Scavenging in vertebrates}

\author{Adam Kane, \and Kevin Healy, \and Thomas Guillerme \and Graeme Ruxton, \and \& Andrew Jackson.}

\affiliations{
\item A. Kane (\url{adam.kane@ucc.ie}), University College Cork, Cooperage Building\, School of Biological 
Earth and Environmental Sciences\,Cork, Ireland.
\item K. Healy and A. Jackson, Trinity College Dublin, Department of Zoology;
School of Natural Sciences, Dublin Ireland.
\item T. Guillerme, Imperial College London, Silwood Park Campus, Department of Life Sciences, Buckhurst Road, Ascot SL5 7PY, UK.
%TG: hehehe!
\item G. Ruxton, School of Biology, Sir Harold Mitchell Building, Greenside Place, St Andrews, KY16 9TH,
United Kingdom
}

\nwords{9999}


\begin{document}


\maketitle


\begin{abstract}
  Scavengers existed in the past and they exist now. 
  Often under appreciated. 
  Three main habitat types considered: land, air and sea. 
  Different drivers in these areas. 
  Review looks at these 
\end{abstract}


\newpage


\section*{Introduction}
%TG: it might be better to reorganise the intro as:
%§1 - scavenging is understudied because it's hard to say who's a scavenger
%§2 - it's even more problematic in the fossil record but people have tried some indirect approaches
%§3 - good thing is that we can apply these indirect approaches to extant systems
%§4 - in this review we discuss some stuff.
%TG: or something along these lines, obviously flexible!
Historically, scavengers have not been viewed as the most charismatic of animals. This may go some way to explaining the gap in our knowledge of the prevalence of this behaviour. Consider Professor Sanborn Tenney writing in 1877 for The American Naturalist who had this to say about one well known group, “Prominent among the mammalian scavengers are the hyenas, the ugliest in their general appearance of all the flesh eaters.” He contrasts these with “nobler kinds” of carnivores such as lions and tigers. Even aside from our own subjective biases, scavenging is a difficult behaviour to detect after the fact. Without catching a carnivore in the act of killing we are left to infer how the prey was killed. Some simple heuristics can inform us, for instance, in cases where the prey item was simply too large to have been killed by the ostensible predator \citep{pobiner2008paleoecological}. But clearly, a scavenger doesn’t only feed on animals too big for it to have hunted. The obvious lack of direct behavioural data compounds the difficulty of discerning scavenging among extinct forms. Indeed, a single species of dinosaur notwithstanding, a synthesis describing the natural history of scavengers is absent from the literature. Fortunately, research in this area is on the rise. As a result we are now beginning to realise the extent of this behaviour such that, “in some ecosystems, vertebrates have been documented to assimilate as much as 90\% of the available carrion" \citep{benbow2015introduction}. Even Tenney’s noble big cats are now known to take in a significant portion of carrion in their diet where some lion populations get over 50\% of their meat from carcasses. A suite of methods have been used to discern the most suitable morphologies, physiologies and environments for a scavenging lifestyle to prosper. It is our aim in this review to employ these methods to gain an understanding of scavengers past and present. 

The chief hurdle to scavenging is finding a sufficient quantity of food, the occurence of which is difficult to predict in space and time. The idea of scrounging from predator kills is undermined from studies showing that in the  majority of ecosystems more animals die from disease and starvation than predation \citep{benbow2015introduction}. Thus, any animal existing as a scavenger must maximise its detection capabilities and minimise its locomotory costs \citep{ruxton2004obligate}. The habitat must also be productive enough to sustain an animal biomass that will eventually produce carcasses. It is recognised that scavengers keep energy flows at a higher trophic level in food webs than decomposers because they consume relatively more carrion \citep{devault2003scavenging}. 

\section*{Aerial Scavengers}

Vultures represent the best known scavengers on Earth. These birds consist of two convergent groups, from the old and the new world and represent the only example of obligate vertebrate scavengers today. Given their unique position, they have been extensively studied to determine what adaptations they possess that allows them to so flourish in this niche. As such, we can begin by exploring the adapdations and the environments of vultures to draw comparisons with other scavenging species and \textit{their} environments. 

Species capable of flight have effectively added an extra spatial dimension, i.e. the vertical component, to their sensory environment over land animals. This allows them to look down on a landscape where they are unencumbered by obstacles that would obstruct the view of a terrestrial scavenger. Such an ability has obvious benefits in detecting carrion. Vultures are known to have impressive visual acuity with one estimate indicating Lappet-faced Vultures (\textit{Torgos tracheliotus}) are capable of detecting a 2 metre carcass over 10 km away \citep{spiegel2013factors}. We know that many birds exist as facultative scavengers; storks, eagles, corvids, are all known to take substantial quantities of carrion in their diet. And eagles in particular are known to have highly developed visual abilities. It follows from this that the evolution of flight allowed aerial animals to detect far more carrion than their terrestrial counterparts.

Moreover, having a panoramic view means being able to gather a wealth of information from other foragers, be they conspecifics or other species. Again, returing to vultures, the genus \textit{Gyps} consists of highly social and colonially nesting species. These behaviours allow them forage far more efficiently because one bird can scrounge information on the location of food from another successful forager. 

Flight is also cheaper means of locomotion than running \citep{tucker1975energetic}. This advantage can be extended further in larger species by engaging in soaring instead of flapping flight, which is even cheaper still (approximately twice BMR) \citep{hedenstrom1993migration}. The advantages this confers are clear from the information we have on the enormous foraging ranges of some seabirds and accipiters. Clearly, it would be pointless to have incredible detection abilites and not have a cost efficient movement to benefit from it. 

Avian flight originates in the Jurassic Period, conincident with the fossils of \textit{Archaeopteryx lithographica} so many of these benefits would have been realised from that point on for carnivorous birds. However, vertebrate flight is much older than this where pterosaurs predate bird origins by a considerable margin in the Triassic Period. Scavenging in this diverse group has been hypothesied many times before. Certain clades of these animals could reach enormous sizes (e.g. Azhdarchids with wingspans of 11 metres) and look to have engaged in soaring flight.  Although \cite{witton2008reappraisal} argued that neck inflexibility and straight, rather than hooked jaw morphology points against their existing as \textit{obligate} scavengers, Azhdarchid terrestrial proficency indicates they would have been comfortable foraging on the ground. Indeed, extant Marabou Storks have a comparable morphology and are noted facultative scavengers so it is reasonable to believe that certain pterosaurs behaved similarly.   

Large body size confers substantial dominance benefits \citep{ruxton2004obligate}. Thus, we would expect scavengers to have this trait selected for even in the case of weight-constrained fliers. Cinereous Vultures (\textit{Aegypius monachus}) and condors (\textit{Vultur gryphus}, \textit{Gymnogyps californianus}) all have body masses that can exceed 10 kg and represent some of the heaviest bird species capable of flight \citep{ferguson2001raptors,donazar2002effects}. And as we have noted the Azhdarchid pterosaurs were far bigger again, with estimated body masses of around 80 kg. 
 
The only other vertebrate group capable of powered flight are the bats where scavenging has not been recorded to our knowledge. Their visual acuity is famously poor and echolocation does not lend itself to discovering immobile carrion. Their small size and poor terrestrial ability would also count against them at a carcass. The bat fossil record is notoriously poor owing to their fragile skeletons so we are unable to determine if extinct species were more suited to this lifestyle. 

Vertebrate scavengers in general are responsible for the dispersal of nutrients. Consider the diversity of animals that can end up feeding at the carcass of an elephant. Here we have an incredibly dense and nutrient rich patch that ends up being distributed widely. Thus, in an ecological context, the evolution of flight coupled with the ability to scavenge resulted in a world with a far more widely distributed nutrient landscape. In the absence of vertebrate scavengers, invertebrates and microorganisms would consume the carcass in-situ or at least distribute the constituent nutrients over a much shorter range. 

\section*{Terrestrial Scavengers}
A simplificaiton of terrestrial scavengers is one of them existing in a two-dimensional plane while foraging for carrion directly. They can detect carcasses at a range that is defined by the radius of their sensory organs, usually the visual and olfactory senses. As a consequence, they have a much more restricted view of the landscape than do aerial foragers. No contemporary terrestrial vertebrate exists as an obligate scavenger but most if not all are facultative scavengers to some extent. \cite{ruxton2004obligate} offer a reason for this in that the traits that allow for vultures to exist as scavengers undermined their ability to hunt but that the same forces have not prevented mammals from doing so. The same authors in a theoretical study do concede that "a 1 tonne mammal or reptile, in an ecosystem yielding carrion at densities similar to the current Serengeti, could have met its energy requirements if it could detect carrion over a distance of the order of 400–500 m."\citep{ruxton2004obligate}. 

Terrestrial scavenging in the mammals is probably best known in an African context where hyenas, jackals and lions all take sizable proportions of carrion in their diet. In the spotted hyena (\textit{Crocuta crocuta}), striped hyena (\textit{Hyaena hyaena}) and brown hyena (\textit{Hyaena brunnea}) it can be as high as 99\% \citep{benbow2015introduction}. Therefore, we can again use these species as our efficient terrestrial scavengers to compare with other forms. 

Similar to vultures they have well developed sensory organs, particularly in olfaction whereby they can detect a rotting carcass 2 km downwind. They have a characteristic "rocking horse gait"  which allows them to cover great distances efficiently. The bone crushing ability of hyenas reveals another useful scavenger trait. Since carrion is not dispatched directly, often the most easily accessible and choicest components of the carcass will be missing or, if present, will be fought over. Being able to extract nutrients from remnants gives the scavenger a great advantage. Osteophagy is known across a range of terrestrial carnivores. Some fat-rich mammalian bones have an energy density (6.7 kJ/g) comparable with that of muscle tissue, making skeletal remains an enticing resource \citep{brown1989study}. This ability reached its zenith among hyenas with the evolution of the 110 kg \textit{Pachycrocuta brevirostris} during the Pliocene \citep{palmqvist2011giant}. The ability to process bone means a carcass fed on by hyenas will be reduced to nothing, whereas the skeleton will remain in carrion attended by species restriced to feeding on the flesh. 

Many of these adapations to scavenging are found in the other major extant terrestrial mammalian carnviores, the bears, dogs and cats to a greater or lesser extent. Though the specific mix of features realised in hyenas suggest this is the model organism for terrestrial scavenging among mammals in the past. Indeed, the bone-crushing dogs that evolved during the Oligocene (subfamily Borophaginae) have been compared to hyenas in terms of their feeding ecology \citep{van2003chapter,martin2016pursuit}. Interestingly such comparisons have given insight into the feeding ecology of early hominins who, for instance, had the ability to craft tools for breaking open bones \citep{hone2010feeding,ARCM:ARCM12084}. 

By contrast, a successful reptilian scavenger requires a far different set of adapations. Modern forms are ectothermic, limiting their activity periods. This is exacerbated by the sprawling gait seen in lizards which results in Carrier's Constraint such that the animal can't move and breathe at the same time because the lateral movements impedes its lungs. This manifests itself in aspects such as maximum sustainable speed where an equivalent mammal has a six to seven fold increase \citep{ruben1995evolution}. A lower metabolism does give reptiles an advantage however, in that over the course of a year their food requirements can be 30 times smaller than an endotherm of equal size \citep{Nagy1621}. Any adaptations that reduce energetic costs are likely to be selected in scavengers. \cite{devault2002scavenging} suggest this is an avenue for scavenging in snakes because they "exhibit  exceedingly  low  maintenance  metabolisms,  and most  can  survive  on  a  few  scant  feedings per year. It  is, therefore, possible for snakes to rely largely  on  infrequent,  less  energy-rich  meals." In the same review the authors found occurrences of scavenging spread across five families of snakes and stated that this behaviour is "far more common than currently acknowledged."\citep{devault2002scavenging}. 

Unsurprisingly, given their enduring appeal, the prevalence of scavenging has been explored in the carnivorous, theropod dinosaurs. These animals ranged from the chicken-sized to the whale-sized all of which were bipedal. They are quite alien to anything we know today which restricts our ability to understand their ecology far more so than extinct mammals \citep{weishampel2004dinosauria}. Of relevance, are the questions that still persist about their metabolism \citep{grady2014evidence} and sensory perception. We do know that they walked with the erect gait of mammals or birds rather than the sprawling gait of lizards and that they were most likely facultative scavengers \citep{depalma2013physical}. Much work has focused on the existence of the behaviour in \textit{Tyrannosaurus rex} \citep{ruxton2003could,carbone2011intra} but a recent energetics study investigated the likely prevelance of scavenging across a range of body sizes. In it the authors demonstrated that species of intermediate body masses (approx. 500 kg) would have gained the most benefit from scavenging. This was the result of gut capacity limitations and the effects of competition at the carcass. At the larger extreme this owes to the fact that gut capacity doesn't scale isometrically with body mass so the benefits of greater mass level off; there's only so much food an individual can consume at a single sitting. For the smaller species, larger competitors would have prevented their access to carrion. 

As we discussed for the case of Cenozoic carnivores, osteophagy could be extremely beneficial to a scavenger. 
In Mesozoic systems some extremely large theropod dinosaurs had a morphology which suggests an ability to process bone e.g. the robust skull and dentition of \textit{T rex}. There is direct evidence that \textit{T.rex} did this in the form of distinctive wear marks on its tooth apices \citep{farlow1994wear,schubert2005wear} and the presence of bone fragments in its coprolites \citep{chin1998king}. The animal also had an enormous bite force, with one estimate putting it at 57000 Newtons \citep{bates2012estimating}. This is noted as being powerful enough to break open skeletal material \citep{rayfield2001cranial}. Osteophagy may have been even more viable during this era because the body mass distribution of herbviores tended to be skewed towards larger sizes \citep{10.1371/journal.pone.0051925}. When we couple this with the fact that skeletal mass scales greater than linearly with body mass \citep{prange1979scaling} there would have been a lot of bone material to consume in the environment provided an animal had the biology to process it  \citep{chure1997one}. 

Evidence of vertebrate scavenging dates back to the early Permian approximately 300 MYA \citep{reisz2006articulated}.

\section*{Aquatic Scavengers} 
An aquatic environment presents challenges for direct observational studies and so, similar to the approaches involving extinct species, much work has approached the question of scavenging propensity from an energetics perspective. The existence of an obligate scavenger in a marine setting is uncertain \citep{britton1994marine,smith2003ecology,ruxton2004energetic,ruxton2005searching}. Carrion in this environment is produced by marine organisms when their carcasses descend to the sea floor. In this low-light environment detection distances are far lower (< 100 m) than they would be in the air. As such, animals detect resources through chemo- and mechanoreception more so than through vision \citep{ruxton2004energetic}. However, water is a medium that is conducive to low-cost movement \citep{tucker1975energetic} and so may be able to support an obligate scavenging fish \citep{ruxton2004energetic,ruxton2005searching}. \cite{benbow2015introduction} do note that "some benthic scavengers (e.g., hagfish: family Myxinidae) rely on necrophagy for a large portion of their diet and may indeed be obligate scavengers". 

Extant aquatic snakes are deemed as having the most suitable physiology and environment for scavenging. A hypothesis put forth by \citep{sazima1990necrofagia} argued that chemical gradients in water would allow for a relatively easier detection of carrion. This gained some support from \cite{devault2002scavenging}, who found a preponderence of aquatic snake species in their review of this behaviour. 

The presence of occasional bounties of carrion in the form of whale falls has led some researchers to investigate if a scavenger could survive by seeking out these remains exclusively. \cite{ruxton2005searching} argued that although this is energetically feasible it's ecologically unlikely. Any animal that could seek out such whale carcasses is unlikely to have ignored other types of carrion. Although no aquatic species have ever exceeded the size of whales, some enormous animals have evolved in this environment before the evolution of whales, including \textit{Leedsichthys}, a bony fish from the Jurassic Period, that weighed in excess of 20 tonnes. Thus, the energetic feasiblity of a marine scavenger has a long history.

As with the aerial and terrestrial enviornments we have evidence of facultative scavenging among extinct, aquatic species. For example, the remains of a mosasaur and a terrestrial hadrosaur were discovered with embedded teeth from a Cretaceous shark \textit{Squalicorax} \citep{schwimmer1997scavenging}. As well as a likely instance of scavenging between a 4-million-year-old white shark (\textit{Carcharodon}) and mysticete whale from Peru \citep{ehret2009caught}.



 %\citep{sazima1990necrofagia} predicted that aquatic or semi-aquatic piscivorous snakes  would  scavenge  more  frequently than other species. Moreover, \cite{sazima1990necrofagia}  suggested that water currents that induce  aggregations of carrion heighten the probability of carrion detection by these snakes (see also \cite{savitzky1992laboratory}).  Also,  chemical  gradients may  be  more  uniform  and  give  more  de-pendable  directional information in water \citep{sazima1990necrofagia}." \citep{devault2002scavenging} 

%"Apart from being an important water-conservation strategy, nocturnal behavior may have evolved in the Hyaenidae as a means of reducing competition with the other dominant scavengers in African ecosystems, the vultures, which are exclusivly diurnal (Houston 1979)" Carnivore Behavior, Ecology, and Evolution By John L. Gittleman

% "Thus, on an annual basis, the FMR (and food requirements) of an endotherm are likely to be around 30 times greater than those for an equivalent-sized ectotherm" 


%"Maximal sustainable speeds of mammals are six- to sevenfold those of lizards of equal size,"

%Scavenging is a widespread behaviour among vertebrates where most if not all carnivores act as facultative scavengers.
%TG: or something more along the lines as:
%Most if not all carnivorous vertebrates are facultative scavenging behaviour to some extant.
%It is recognised that scavengers have an important role in keeping energy flows at a higher trophic level in food webs than decomposers because they consume relatively more carrion \citep{devault2003scavenging}. 
%Scavengers also provide useful ecosystem services by acting as barriers to the spread of disease by quickly consuming rotting carcasses which have often died from illness \citep{ogada2012dropping}.
%(Since we are intrested in scavanging in th paleo record ecosystem services might not be that relavent, although modulators of disease is still relavent) 
%Despite this, scavengers are a seriously understudied group \citep{sekercioglu2006increasing,selva2007nested,wilson2011scavenging}.  
%\cite{devault2003scavenging} propose that this is due to both human disgust at carrion itself and the difficulty in determining if an ingested prey item was killed or scavenged. 
%The latter point means that studying the natural history of this behaviour is particularly problematic.
%Indeed, even data on the proportion of carrion in the diet of extant species are sorely lacking \citep{benbow2015introduction}.

%"Indeed, in some ecosystems, vertebrates have been documented to assimilate as much as 90\% of the available carrion"\citep{benbow2015introduction}.


%The limitations in studying extant scavenging behaviour is much larger in extinct species and systems with the obvious lack of observational data available.
%TG: or more like this? 
% The limitations in studying scavenging in extant species are even bigger in extinct species and past systems since the obvious lack of available direct behavioural data (CITE) %TG: bet you there's some paper for that
% This means indirect observations in the fossil record and other approaches such as energetics must be used to infer these behaviours. %TG: I'll squiz your Am Nat paper here!
% One avenue to infer scavenging from palaeontological data can be achieved by determining if a prey item was simply too big for the carnivore to have tackled in cases where tooth marks are found \citep{pobiner2008paleoecological}. 
% Comparative analysis can also allow us to ascertain which morphologies and physiologies are likely to have been found in scavenging species in the past \citep{ruxton2004obligate}.
% The development of indirect measures of scavenging in palaeontology can in turn be applied to current scavenging systems that also suffer from a lack of observational data.

%§4
% In this review we collate methods (could this be another way of structuring it, just an idea) and research form palaeontology relating to scavenging behaviour and show that ignore this literature would be a missed opportunity for understanding extant scavenging.
%TG: totally agree. I think we need to clearly know what this review is about: scavenging through time (like a story line of scavenging past and present - a bit boring if I can speak my mind) or how to study scavenging (through time or any other aspect - probably more interesting to more people I guess).
%TG: additionally I find the divisions land/air/sea and Ceno/Meso/Paleozoic are a bit scholar. It might be better to just discuss the different techniques for investigating scavenging and include land/air/sea and Ceno/Meso/Paleo in there no? For example:
%\section{Method 1: direct observation}
%This can be totally doable in extant system for scavengers everywhere.
%It is advantageous because it's direct and reliable observations.
%But it has some limitations such as sending Adam to sit in the sun for ours and is hard to apply in deep sea environements or in any past ecosystems...


%Our review is divided up into three sections, namely the land, air and sea. 
%These are then subdivided into three geological eras the Cenozoic, Mesozoic and Palaeozoic. 
%Each of these environments has a distinct phyiscal character that affects how a species forages for food which has obvious relevance for an animal searching for carrion.  
%However, there is some commonality to a scavenger's environment and the problems in finding food that one would encounter. 
%Notably, the resource environment of a scavenger is a patchily distributed one, because it is difficult to predict both when and where a carcass is produced. 
%As a result of this, any animal existing as a scavenger must maximise its detection capabilities and minimise its locomotory costs \citep{ruxton2004obligate}.
%Exploring which groups are likely to have moved towards these traits and thus existed as sccavengers over palaentological time forms the basis of this work.
%remember the journal is an ecological one at heart so I would always keep in mind about making it useful for ecologists over paelo people. In particular from the website "ECOGRAPHY publishes papers focused on broad spatial and temporal patterns, particularly studies of population and community ecology, macroecology, biogeography, and ecological conservation. Studies in ecological genetics and historical ecology are welcomed in the context of explaining contemporary ecological patterns".

%"Interkingdom Competition among Vertebrates, Invertebrates, and Microbes"\citep{benbow2015introduction}
%"For example, vertebrate scavengers appear to be disadvantaged when humidity and temperature favor microbial and invertebrate reproduction" \citep{benbow2015introduction}

%"However, in general, vertebrate scavenging represents the widest dispersal of nutrients and energy from carcasses as vertebrate movement scales away to the broader landscape" \citep{benbow2015introduction}

 %"Increased kill rates by top predators may represent a little acknowledged marginal cost to the vertebrate community, directly attributable to scavenging activity. Given the impor -tance of top-down effects in many ecosystems, even a minor alteration to predation rates as driven by vertebrate scavengers may cause a significant flux in community composition."\citep{benbow2015introduction}

 %"Cortés-Avizanda et al. (2009) found that the abundance of prey species (i.e., hares—Lepus spp. and squirrels—Sciurus  spp. in this case) decreased in sectors containing a carcass based on evidence from tracks in snow. An interesting hypothesis emerged, in which the scavengers that are recruited to a carcass may have temporarily played the dual role of increasing predator abundance near each carcass"\citep{benbow2015introduction}


%"Historically, the prevalence of scavenging activities has been greatly underestimated. However, upon recognition that (1) in most ecosystems, a large number of animals die from causes other than predation and thus become available to scavengers; (2) most carcasses are scavenged by vertebrates before they are completely decomposed by arthropods and bacteria; and (3) almost all carnivorous animals are facultative scavengers, the importance of scavenging in food webs seems unsurprising" \citep{benbow2015introduction}

%"This dispersion of carrion biomass by vertebrates is especially evident when carrion is initially concentrated spatially. For example, carcasses produced from fishing by-catch (Furness 2003), salmon (Salmonidae) die-offs (Hewson 1995), forest fires (Blanchard and Knight 1990), and single large carcasses (e.g., whales—Cetacea; Smith and Baco 2003) are often visited by multiple scavengers that range widely and therefore transport the nutrients from those carcasses over large distances."\citep{benbow2015introduction}

%"Cross-habitat nutrient transport can produce a variety of important outcomes in recipient systems (e.g., Polis et al. 2004), and scavengers can play a significant role in moving “ecologi -cal subsidies” between habitats. For example, the use of ocean-derived carrion by terrestrial mammals (Rose and Polis 1998) and birds (Schlacher et al. 2013) is extensive and may strongly influence dynamics of coastal food webs."\citep{benbow2015introduction}

% "Markandya et  al. (2008) estimated that the total costs to human health (including rabies cases from dog bites) that resulted from severe vulture declines totaled over $34 billion from 1993 to 2006. Also, Ogada et al. (2012) determined that the exclusion of vultures from large animal carcasses in Kenya resulted in a tripling of carcass decomposition times."\citep{benbow2015introduction}

% "Changes in global climate are also expected to substantially alter the availability of carrion resources in marine ecosystems as surface production of organic material could decline by 50% or more (Smith et al. 2008). Such a drastic reduction in productivity undoubtedly will impact a multitude of ecosystem processes, including the availability and distribution of carrion within marine ecosystems." \citep{benbow2015introduction}




%\subsection*{Cenozoic}
%Among terrestrial African carnivores, hyenas, jackals, lions and leopards all take sizable proportions of carrion in their diet.
%In the case of the spotted hyena (\textit{Crocuta crocuta}) it can be as high as 99\% \citep{benbow2015introduction}. 
%Yet, no contemporary terrestrial vertebrate exists as an obligate scavenger. 
%The selective pressures that push mammals and reptiles towards scavenging do not seem to undermine their ability to hunt, perhaps explaining the absence of obligate scavengers in these groups \citep{ruxton2004obligate}.
%\\There is a long running debate on the tendency of human ancestors to scavenge. 
%Some recent studies have found "that passive scavenging from abandoned larger felid kills could have been a high-yield, though potentially dangerous, foraging strategy for early hominins even without considering within-bone nutrients" \citep{pobiner2015new}. 

%Osteophagy is known across a range of terrestrial carnivores.
%Some fat-rich mammalian bones have an energy density (6.7 kJ/g) comparable with that of muscle tissue, making skeletal remains an enticing resource \citep{brown1989study}. 
% Hyenas have a bite force capable of breaking open bones and 
% In light of this, the skeletal remains of carrion may act as trove of food to carnivores who can access it.  



%"Of all the mammalian carnivores in Africa, brown, striped, and spotted hyenas (hereafter referred to simply as hyenas) derive the largest portion of their diet from scavenging. This is not surprising, as they show many unique adaptations specifically for scavenging. They have a unique body posture and a “rocking-horse” gait (Eloff 1964; Tilson and Henschel 1986; Hofer and East 1993a; Frank 1996) that allows them to cover long distances in search of carrion and prey."\citep{benbow2015introduction}

% A review by \cite{devault2002scavenging} found occurrences of scavenging spread across five families of snakes and states that this behaviour is "far more common than currently acknowledged."



%"\cite{sazima1990necrofagia} predicted that  several components  of  snake  be-havior contribute to scavenging propensity. They  postulated  that  snakes  that  rely  on chemosensory information to acquire prey might forage  for  carrion  more  frequently than  those  whose  modes  of prey acquisition are driven by visual cues. Additionally, \cite{sazima1990necrofagia} suggested that habitat and diet preferences might be indicators of scavenging   propensity   be-cause  they  influence  the  likelihood of a particular species encountering carrion naturally. Based on these criteria, they pre-dicted that aquatic or semi-aquatic pisciv-orous snakes  would  scavenge  more  fre-quently than other species. Moreover, \cite{sazima1990necrofagia}  suggested that water currents that induce  aggrega-tions of carrion heighten the probability of carrion detection by these snakes (see also \cite{savitzky1992laboratory}).  Also,  chemical  gradients may  be  more  uniform  and  give  more  de-pendable  directional information in water \citep{sazima1990necrofagia}." \citep{devault2002scavenging} 



%"Investigations of ophidian metabolic re-quirements unveil additional advantages to carrion  utilization.  Snakes  exhibit  exceed-ingly  low  maintenance  metabolisms,  and most  can  survive  on  a  few  scant  feedings per year. It  is, therefore, possible for snakes to rely large-ly  on  infrequent,  less  energy-rich  meals. Carrion, which is by nature ephemeral and unpredictable, may represent such a food source  to  snakes.  Scavenging  might  allow snakes to meet their low metabolic needs more  easily  and  without  the  costs  associ-ated with subduing prey." \citep{devault2002scavenging} 




%\textit{Allosaurus} tooth marks on a hadrosaur in the Late Jurassic. 
% Late Triassic scavenging on a prosauropod by basal carnivorous archosaurs \citep{hungerbuhler1998taphonomy}.










%\subsection*{Palaeozoic}
% The absence of flying vertebrates in the Palaeozoic may have permitted terrestrial forms to take in a higher proportion of carrion in their diet. 







%\subsection*{Cenozoic}
%A likely instance of scavenging between a 4-million-year-old white shark (\textit{Carcharodon}) and mysticete whale from Peru \citep{ehret2009caught}.
%Bite marks on early Holocene Tursiops truncatus fossils from the North Sea indicate scavenging by rays (Chondrichthyes, Rajidae) \citep{van2009bite}. 
%Possible scavenging on a juvenile fur seal from the Late Neogene \citep{boessenecker2011mammalian}. 

%\subsection*{Mesozoic}

%\subsection*{Palaeozoic}
%Evolution of sharks, known scavengers. 




%\section*{Results}




%\section*{Discussion}



\section*{Acknowledgments}

A lot of people are to thank here.


\newpage


\bibliography{bibfile}



\end{document}
