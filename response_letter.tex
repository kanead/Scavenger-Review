\documentclass{letter}
%\signature{Your name}
%\address{Street \\ City \\ Country}
\begin{document}
\begin{letter}{Company name \\ Street\\ City\\ Country}
\opening{To the editor}

We would like to thank the editor, subject editor and reviewers for their comments. By bringing attention to elements of the manuscript that require more careful communication we believe that you have helped us to significantly improve our manuscript. In this submission we have clarified these areas and added a body of work to bolster the conclusions drawn from our work. Please see below where we consider each of your points in turn. 

Reviewer: 3

Comments to the Author
This is a well-written, enjoyable manuscript that assesses the functional traits associated to past and, mostly, current major vertebrate scavengers. As a result, the authors provide a sort of guide to identify the extent of scavenging behaviour among carnivores in both current and past ecosystems. I have several suggestions that should be considered before publication (please see below).

Major comments:

1. Invertebrates are ignored completely throughout the ms. Please specify in the Introduction that this review focuses on vertebrate scavengers. In this line, I recommend to include “in vertebrates” at the end of the title to be more specific. Also, in the “Competition” section you mention competition with vertebrates and micro-organisms, but not with invertebrates. You should consider them in this section (and in “Facilitation”; see next point).

2. I found difficult to follow the main point of the paper because the ms is not well organized. Following a hierarchical order, I’d define four/five main sections: “Introduction”, “The Challenges of Scavenging” (this could be a subsection within the previous section), “Encounter Rate” (with four different subsections: “Metabolism”, “Locomotion”, “Sensory Detection” and “Carcass Availability”), “Handling Time” (which would include “Food Processing”) and “Conclusion”. \\ 

Moreover, “Competition” and “Facilitation” (I strongly encourage you to consider not only competitive, but also facilitative processes, which are mostly neglected in your review) should be treated as transversal factors that can modulate each of the abovementioned parameters. Thus, I’d include them in a different section, or mention the competitive and facilitative processes related to each parameter within each subsection. Fig. 1 should be re-organized accordingly.

3. The use of the scientific literature on scavenging is deficient, especially for a review paper like this. A number of key references are missing, e.g., Selva \& Fortuna (2007 Proc. R Soc. B 274:1101-1108), Wilson \& Wolkovich (2011 TREE 23:129-135), Beasley et al. (2012 Oikos 121:1021-1026), Cortés-Avizanda et al. (2012 Ecology 93:2570-2579), Cortés-Avizanda et al. (2014 Ecology 95:1799-1808), Moleón et al. (2014 Biol. Rev. 89:1042-1054), Pereira et al. (2014 Mammal Rev. 44:44-55), Périquet et al. (2015 Biol. Rev. 90:1197-1214), Mateo-Tomás et al. (2015 Divers. Distrib. 21:913-924), Moleón \& Sánchez-Zapata (2015 BioScience 65:1003-1010), Moreno-Opo et al. (2016 Behav. Ecol.), Sebastián-González et al. (2016 Ecology 97:95-105). Your literature synthesis might benefit from the reading of these articles, for instance regarding competitive and facilitative interactions among scavengers. Also, the review would be more informative and useful for readers. Next I mention some particular comments regarding the use of the references (I mention others in Minor comments): \\

Page 2, 1st sentence: but see Moleón \& Sánchez-Zapata (2015); many of the most charismatic vertebrates of the world are scavengers. Rather than “scavengers”, what is not charismatic is probably “scavenging”.

We now refer to the behaviour rather than the species as lacking charisma. 

 Page 2, 13th line: Moleón \& Sánchez-Zapata (2015) may be more appropriate than Koenig (2006) because the former is more general and not restricted to vultures.
 Page 2, lines 17th-18th: see Pereira et al. (2014) and Périquet et al. (2015) for reviews.

These references have been added. 

 Page 3, bottom paragraph, bottom line: this is not true, as scavenging may indirectly affect herbivore populations and thus carrion availability (see Moleón et al. 2014 Biol. Rev.).

The potential for scavengers to affect carrion availability has been acknowledged. 

 Page 5, bottom paragraph, 1st sentence: Mateo-Tomás et al. (2015) is more general than Kendall (2013).
 Page 8, 1st and 2nd sentences: see Pereira et al. (2014) and Périquet et al. (2015).
 Page 8, 3rd sentence: you may also cite DeVault et al. (2003) and Pereira et al. (2014).

These references have been added. 

 Page 8, 8th-17th lines: please reword; wild dogs hardly scavenge, while leopards and lions are frequent scavengers (see Pereira et al. 2014).

We have reworded this section to make the more general point that ambush predators can rely more on hunting than can cursorial species, a point explicitly stated in the Pereira et al. (2014) article, which we now cite. 

 Page 11, 1st sentence: see Kane et al. (2014) and Moleón et al. (2014 Biol. Rev.) for inter-specific interactions.
 Page 11, 4th sentence: Kane et al. (2014) is not appropriate here because it is about inter-specific interactions; use Cortés-Avizanda et al. (2014) instead.
 Page 14, 2nd paragraph, 1st sentence: see also Moleón et al. (2014 Biol. Rev.) and Pereira et al. (2014).
 Page 14, 2nd paragraph, 2nd sentence: see better Pereira et al. (2014) and Périquet et al. (2015). 
 Page 14, bottom paragraph, 3rd sentence: you may also consider Moreno-Opo et al. (2016).
 Page 15, 2nd paragraph, 2nd sentence: see also Moleón et al. (2014 Biol. Rev.) and Pereira et al. (2014).

All of these references have been added. 

4. You should include more discussion about the importance of carcass size throughout the ms, as it has a strong influence on scavenging patterns and interactions among vertebrates and between vertebrates and smaller carrion consumers (e.g., see Moleón et al. 2015).

Minor comments:

GENERAL

1. Please provide line numbers to facilitate the reviewing process.

Line numbers have now been included. 

ABSTRACT

2. You could remove “, the first to our knowledge”.

This has been removed.

3. Also, you could mention some relevant specific findings of your synthesis.

THE CHALLENGES OF SCAVENGING

4. Page 3, 1st sentence: “often difficult to predict” compared to what? Please specify. There are many examples in which carrion is highly predictable, e.g., during salmon spawning (see also Pereira et al. 2014).

ENCOUNTER RATE

5. Page 3, bottom paragraph, 3rd line: change “Alternatively” to “Also”?

This has been changed. 

METABOLISM

6. Page 4, 1st sentence: “the sporadic nature of carrion”; please explain.

LOCOMOTION

7. Page 4, bottom paragraph, 1st sentence: change “inherent” to “relative”.

This has been changed. 

8. Page 4, bottom paragraph, 3rd sentence: remove “, paradoxically,”.

This has been deleted. 

9. Page 5, 2nd paragraph, 1st sentence: change “appear” to “appeared”. Also, specify Mya for the Cretaceous period.

This has been changed. 

10. Page 9, top paragraph: please be more cautious in your assertion, as cooperation skills and tools of early hominins probably allowed them to defend and usurp carcasses from more powerful scavengers.

We have changed this paragraph and take into account the authors' reference to tool use and sociality in hominins. The point is that even with these advantages, running to a carcass over long distances still wasn't energetically sensible. 

DETECTION

11. Page 11, top and 2nd paragraph, 1st sentence: this is about facilitation. As said above, facilitative processes may affect other parameters as well, not only the ability to detect carrion.

12. Page 11, last sentence: explain why wind speed affects the olfactory capacity of an aquatic species.

PREY AVAILABILITY

13. I’d change this title to “Carcass Availability”, which is more specific.

This has been changed. 

14. Among the different components of carcass availability, e.g., abundance, duration, predictability and accessibility, in this section you only talk about abundance and duration. You should also discuss about carcass predictability (e.g., see Cortés-Avizanda et al. 2012) and carcass accessibility (e.g., see DeVault \& Krochmal 2002).

15. Page 12, 2nd paragraph, 1st sentence: as said before, this is not true, given that scavenging may indirectly affect herbivore populations and thus carrion availability (see Moleón et al. 2014 Biol. Rev.).

We have removed the part of the sentence where we mistakenly suggest that scavengers cannot affect carrion availability. 

16. Page 13, 3rd and 4th sentences: whaling has exerted also a huge negative impact on scavengers by depleting whale populations. If you are going to talk about human impacts, maybe is better to do it at the end of the ms, perhaps in a new section in which you explore which scavenging traits could be more successful in a human dominated world.

This has been removed. 

17. You should also take into account facilitative processes. For instance, some scavengers are able to access the interior of carcasses only once other, more powerful species, has opened the skin (see Moleón et al. 2014 Biol. Rev.).

18. Page 14, bottom paragraph, 2nd sentence: I’d change it to “This is true for many vultures and other major avian scavengers such as albatrosses who all have […]”.

We have reworded the sentence as suggested. 

FOOD PROCESSING

19. Page 16, 2nd paragraph, 6th sentence: insert “Within mammals,” before “This ability”.

This has been added. 

20. Page 17, 2nd paragraph, 4th sentence: do you refer to species or individuals? Higher species diversity doesn’t mean more individuals.

The reference of Pobiner (2015) states that there were more individuals and we have changed the sentence to reflect this. 

21. Page 18, 1st sentence: change “enabling” to “enable”.

This has been changed. 

22. Page 18, 3rd sentence: “thus”? I don’t understand this sentence; could you please reword it?

We have reworded this sentence. The idea is that early birds didn't have beaks and so could more easily eat meat. This is in contrast to modern birds where a beak may be a hindrance.

23. Page 18, 5th sentence: “The mix of strong and weak features in their skull morphology”; please explain it.

CONCLUSION

24. Page 18, bottom paragraph, 3rd sentence: change “technique” to “approach”?

This has been changed. 

FIGURES

25. Fig. 2: you may insert the name and Mya range of each period in the X axis rather than in the figure legend. Also, I’d indicate “aerial”, “terrestrial” and “aquatic” in the Y axis rather than in the legend.


Hope this helps,
Marcos Moleón

\bigskip
Yours Sincerely,

Adam Kane and co-authors 
%\closing{Yours Faithfully,}
%\ps{P.S. Here goes your ps.}
%\encl{Enclosures.}
\end{letter}
\end{document}