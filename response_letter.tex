\documentclass[12pt,letterpaper]{article}
\usepackage{url}
\usepackage{color}
%\signature{Your name}
%\address{Cooperage Building \\ North Mall Campus \\ University College Cork \\ Cork \\ Ireland}
\begin{document}
%\begin{letter}{Company name \\ Street\\ City\\ Country}
Dear Miguel Ara\'{u}jo,

\bigskip

We would like to thank the editor, subject editor and reviewers for their comments. By bringing attention to elements of the manuscript that require more careful communication we believe that you have helped us to significantly improve our manuscript. In this submission we have clarified these areas and added a body of work to bolster the conclusions drawn from our work. Please see below where we consider each of your points in turn.
For improved clarity in this document and in the manuscript, we have highlighted our responses in \textcolor{blue}{blue}.

\section{Recommendation by the Subject Editor (anonymous):}
Three experienced reviews have evaluated your manuscript.
Two of them are extremely critical of several aspects related to to the general handcraft and novelly of the review.
The third reviewer is more positive, while making substantial recommendations for improvement of the manuscript.
Should you be able to address all of the reviewers comments in a much revised version of the manuscript I would happy to reconsider the manuscript for consideration in Ecography and the E4 Award.
This is not a provisional acceptance, since the manuscript would need to be revised again.

\smallskip

\textcolor{blue}{As we note below, we feel our earlier manuscript did not capture the hierarchy we had in mind.
This updated version is more explicit in terms of the structure where we now highlight the sections and subsections.
Regarding the novelty of our manuscript, we have changed our introduction which was regarded as too similar to deValult et al. (2003) and argue our case for what we consider inevitable overlap with Beasley et al. (2015).} 
%KH I feel this is a bit weak. While we aknoledge the envitable overlap with Beasely et al we more carefully popint towards how this builds on out qualitative frameworks of how our understanding of extant scavangign can be exxtended to extinct species.

\section{Reviewer: 1}
This paper attempts to synthesize literature on traits that enable scavenging behavior in extant and extinct vertebrates.
The paper is mostly well written, and most (but not all) pertinent literature sources are included.

\textcolor{blue}{Thank you}

I do worry about the lack of novelty of the manuscript, and what I perceived as haphazard organization. 

\textcolor{red}{We feel we have now more clearly outlined the structure of the paper by including an updated figure one explicitly showing the breakdown of the chapters and their link to foraging ecology. This also more clearly highlights the novel aspect of our review, of building a qualitative framework base on foraging ecology to identify the potential levels of scavenging in extinct species and systems and pointing towards possible methods to better focus on certain systems, such as energetic models.}

Here are some specific issues that should be addressed in my opinion:

\begin{enumerate}
\item{Abstract: Scavenging can be difficult to observe, but probably no more so than predation.
Also, using experimental carcasses, scavenging is actually quite easy to observe.}

\textcolor{blue}{We emphasise that food webs have tended to lump all carnivory into predation and it is with extinct species that we have no observational evidence.}
 \textcolor{red}{We now note measures often used in dietary analysis, such as stomach contents and stable isotope analysis, are unable to distinguish between scavenging and predation. We emphasis that the subsequent need for extensive behavioral data has led to food webs have tending to lump all carnivory into predation and this is particularly likely with extinct species. (lines 53-56)}


\item{Abstract: You should point out that your ``scale of scavenging'' is only qualitative; you made no attempt to quantify such traits.}

\textcolor{blue}{We now emphasise that our scale is qualitative here and throughout the manuscript.\textcolor{red}{ in lines (----)} 

\item{Introduction: Much of this section seems to follow DeVault's (2003) review paper on scavenging—in fact, there is almost nothing new here that is not covered in that earlier review.}

\textcolor{blue}{We have rewritten the introduction to more accurately reflect the aim of this review i.e. that the scavenging gradient allows us to infer scavenging in the fossil record as well as in modern species. In making this revision we have reduced the \textcolor{red}{inevitable} overlap with De Vault (2003) considerably.} %TG: 

\item{Introduction: You mention that you will use ``other methods in order to discern the most suitable morphologies...''  but you don't say what they are. What other methods are you talking about?}

\textcolor{blue}{We now specify the approaches we have in mind as \textit{``e.g. energetics models, comparative anatomy, palaeontology etc.''}} \textcolor{red}{on lines 22-24 with references referring to these methods.}




\item{I think that early on in the paper you need to define obligate and facultative scavengers and then throughout the paper, specify which you are referring to. For example, in the first sentence of the Detection section, you talk about ``known scavengers'', but it is unclear if you really mean obligate scavengers. Nearly all carnivorous vertebrates are scavengers at least to some extent, so you need to elaborate on what you mean when you talk about ``scavengers''.}

\textcolor{blue} {We make the following point in our introduction: \textit{``Indeed, there is no discrete divide between predators and scavengers but rather a continuous gradient. Even vultures, the canonical example of obligate scavengers, can hunt.''} We have also been clearer on this point throughout the paper by removing references to this division \textcolor{red}{and instead referring to the gradient of scavenging outlined in the introduction}.}

\item{Last sentence of Introduction: Again, you propose a ``scale of scavenging'', but without quantification, it is unclear how useful this ``scale'' is.}

\textcolor{blue}{As mentioned in the point above the concept of a scale, and not a false caterogisation, is central to our paper. We have now clarified one of the advantages of such a scale at the end of the introduction: \textit{``We argue that such a scale could be valuable when behaviour can not be observed directly, e.g. for extinct vertebrates.''}}
%KH I re-worded this sentance a little bit

\item{Page 6, 12: In your discussion of aquatic scavengers, you might want to refer to Beasley et al. (2012; Oikos 121:1021-1026).}

\textcolor{blue}{We have included this reference and explicitly discuss how differences in the physical properties of water Vs air can affect scavenging \textcolor{blue}{through carcass availability on pages 14 and 15}: \textit{``The physical differences between water and air mean carcass availability is radically different between these environments (Beasley et al. 2012). For one, carcasses get moved around by water which results in a more diffuse signal being produced for would-be scavengers. Carcasses also tend to sink in water where they are no longer accessible to pelagic scavengers (Beasley et al. 2012). Research has shown that an animal need only travel between 16 to 36 km to encounter a fresh whale carcass  (Mole\'{o}n and Sánchez-Zapata 2015). The phenomenon of occasional bounties of carrion in the form of these whale falls has led some researchers to investigate if an obligate scavenger could survive by seeking out these remains exclusively.''}}

\item{Page 6: Why is jaw morphology likely to define scavengers?  Please elaborate.}

\textcolor{blue}{We now avoid discussion of jaw morphology at this early point in the manuscript, and later develop fully our ideas in section 3.3 on how feeding anatomy can be related to propensity to exploit scavenging opportunities}
%TG: Note that there's no change in the text here. Just a lecture on jaw morphologies. We might want to elaborate a bit in the manuscript. Agree we never actually answer his question.

\item{Page 9: It is true that carrion can comprise a large percentage of a hyena's diet, but your text seems misleading. Most studies have shown that most of a hyena's diet is from predation. For example, see Cooper et al. (1999; Afr. J. Ecol. 37:149–160) and Gasaway et al. (1991; Afr. J. Ecol. 29: 64–75).}

\textcolor{blue}{We have clarified this point about hyenas to say it is their ability to take large proportions of carrion under certain circumstances that makes them a useful focal species: \textit{``The particular ability of hyenas to subsist on high proportions of carrion means we can use them as examples of efficient terrestrial scavengers to compare with other forms.''}}

\item{Your section on Handling Time is so short that I think it should be cut or combined with another section.}

\textcolor{blue}{The hierarchy of our paper wasn't clear as we intended in the first version but thanks to the review process we have now clarified that this is the lead in paragraph to a larger section which has subsections e.g. food processing (see also updated figure 1).} 

\end{enumerate}

\section{Reviewer: 2}
This article is a review of the literature what the authors term a ``natural history'' of scavenging.
They have approached the topic from the perspective of ecological variables (and other variables that condition them) in optimal foraging theory including encounter rate and handling time, but also metabolism, locomotion (e.g soaring and swimming), detection, prey availability, competition, and food processing.
They describe many useful examples of the behavior and ecology of obligate and facultative scavengers.
But I think they often go to far in their assumptions in trying to weave these examples together in a synthetic way.
Some examples are inferring scrounging behaviors in flying pterosaurs (pace 11), assuming that hominins used vultures to identify scavenging opportunities (page 11), surmising that scavenging evolved with the earliest tetrapods based on ``potential carrion unexploited by marine vertebrates'' (page 12).
There is a lot of ``these researchers have suggested X'' and then taking that as a given fact.

\bigskip
\textcolor{blue}{One of the central points of our paper is to compare the biology of extant scavengers with extinct species in order to determine if they were suited to scavenging.
And as we discuss in our introduction, because we can't observe the behaviour of extinct species directly we are left to make assumptions about their behaviour. 
Rather than taking them as fact we offer these as possible behaviours that the extinct species engaged in. \textcolor{red}{Our paper proposes these sets of possible scenarios of scavenging to allow for more quantitative tools, such a energetic models, to rule out or support these each scenario leading to a more refined understanding of scavenging in extinct species.}
We have also added extra references in support of our claims e.g. \textit {``Thus, given pterosaurs seem to have cohabited in large numbers (Witton 2013), and the theoretical benefits this can have for social foraging in birds (Dermody et al. 2011), it seems probable that scrounging behaviours were seen in the flying pterosaurs as well.''}}


% All of the examples we use are backed up by references from previous work.
% But this is hardly unique to our work.

\bigskip

While the authors distinguish between obligate and facultative scavenging, they tend to lump them together when it comes to constructing their arguments.
In the end, is the argument about the former, or the latter? Is the real interest in whether species scavenge at all, what proportion of their resources are obtained by scavenging, or what the conditions are that produce facultative or obligate scavengers? 

\bigskip


\textcolor{blue} {This was also raised by reviewer 1. and we thank both reviewers for pointing out this aspect which was unclear in the original manuscript.
We now make the following point in our introduction: \textit{``Indeed, there is no discrete divide between predators and scavengers but rather a continuous gradient. Even vultures, the canonical example of obligate scavengers, can hunt.} We have tried to be clearer throughout the paper as well. Indeed, this is a key point of our review paper: to change our view of scavenging from the classic dichotomy of obligate and facultative scavengers, to a sliding scale depending on the species and the ecosystem.''}}
 \textcolor{red}{We have also been clearer on this point throughout the paper by reducing references to this division between obligate and facultative scavengers were appropriate and instead referring to the gradient of scavenging outlined in the introduction}.
%Is this point true
\bigskip


They also seem to ignore flexibility in species' behaviors.
For instance, while spotted hyenas are on their highly scavenging side of their Figures 1 and 2, in some ecosystems they hunt significant proportion of their prey.
Similarly, lions can take a large proportion of their prey as scavenged, as the authors note.
But to suggest, as they do in their abstract, that they can apply their scale of scavenging to any species at any time to judge how important scavenging was in its diet is frankly completely ignoring behavioral variability over time and space. 
\bigskip

\textcolor{blue}{It is undoubtedly true that scavengers can be variable in the proportion of carrion they consume but we would argue that species like hyenas have certain characteristics that make this capacity to scavenge  a high proportion of their diet possible.
In contrast, cheetahs do not have a biology that lends itself to scavenging in the wild, due to poor competitive ability etc.
We appreciate that there is a continuum across and among species which we specifically note in this revised version \textcolor{red}{(page lines xxx)}. 
As we note in the above comment, a combination of a species' traits, and the state of the ecosystem they find themselves in will determine their scale of scavenging. 
Search rates can be altered by environmmetal or ecological conditions as well as intrinsic traits of the forager. 
So, far from invalidating our idea it reinforces it.
Moreover, researchers working in palaeobiology often try to reconstruct the behaviours of extinct species, \textcolor{red}{which themselves are likely to be extremely variable}; we are following this approach when we make inferences about ancient systems were we \textcolor{red}{are interested in determining whether a species has the capacity to scavenge like a Hyena or is more restricted in this ability such as a cheetah}}
\bigskip

It is substantially more difficult with fossil species for which we may know far fewer of the variables they list in Figure 1, in a nuanced way.
Figure 2 does not display the diversity of scavengers through time, it displays examples of species that display observed or inferred scavenging behaviors.
Diversity has specific ecological definition(s) that are being ignored here.
\bigskip

\textcolor{red}{We agree it is more difficult to infer behavior with lower quality data, however by mapping out the variables that are likely to driver high levels of scavenging, the absence of nuanced measures will highlight the extra caution needed when inferring the existence of scavenging in such a circumstance}
\textcolor{blue}{Figure 2 has been removed and figure 1 has been expanded.}

%TG: They make a point here (although their diversity definition rant is completely out of topic). Should we just drop figure 2 and use it as an abstract/press release picture? AK Agreed 


\bigskip

If the authors chose to resubmit, I would suggest they organize their manuscript based on known information about modern species, known (e.g. bone processing by theropods) and inferred (e.g. scavenging in hominins) information about fossil species, and then try to draw broader conclusions.
They might look to articles about other modern scavenging opportunities such as Capaldo and Peter's study of wildebeest drownings \url{http://www.sciencedirect.com/science/article/pii/S0305440385700398}, and suggestions that sabertooth cats which they claim were unable to feed on bones may have consumed more bone than penecontemporaneous dire wolves \url{https://www.jstor.org/stable/20627167?seq=1#page_scan_tab_contents}. 

\bigskip

\textcolor{blue}{The issue of the structure of our paper was noted by all reviewers who all had suggestions to improve it. 
We have made a significant effort to delineate all of the main sections and subsections so that our frame of optimal foraging is more apparent.
We have also added the two suggested references to our paper.}

\bigskip
In sum, while I think the aim of the article is a very worthwhile one, I think the authors cannot claim to have produced an all-encompassing model that can predict the presence or level of scavenging in any organism - because I think this simply cannot be done.

\textcolor{red}{We agree, and do not intend on our framework to act as an all encompassing model of scavenging but instead as a guide to determine a range of possible sceneries with different levels of scavenging. We highlight throughout the manuscript the requirement to use other methods such as energetic models and comparative anatomy in order to make accurate inferences on behaviour.}

%TG: State somewhere here as a response to the reviewer that we don't develop any model in this paper?

Finally, some of the perspectives of this manuscript overlap substantially with parts of Beasley et al. 2015 (pages 108-112), which is cited numerous times in the manuscript.
\bigskip

\textcolor{blue}{The sections of Beasley et al. (2015) are: 1. the evolution of scavengers, 2. competition between scavengers and 3. environmental effects on scavenger ecology.
These are very broad areas so in a review of this sort some overlap is inevitable.
We feel our inclusion of extinct species and systems as well as the novel framing under optimal foraging theory gives this manuscript its novelty.} \textcolor{red}{We have paid particular attention towards making the structure or our manuscript more clear and hence delineating it from Beasley et al. (2015).
\bigskip

\section{Reviewer: 3}
This is a well-written, enjoyable manuscript that assesses the functional traits associated to past and, mostly, current major vertebrate scavengers.
As a result, the authors provide a sort of guide to identify the extent of scavenging behaviour among carnivores in both current and past ecosystems.
I have several suggestions that should be considered before publication (please see below).

\subsection{Major comments:}

\begin{enumerate}
\item{Invertebrates are ignored completely throughout the ms.
Please specify in the Introduction that this review focuses on vertebrate scavengers.
In this line, I recommend to include ``in vertebrates'' at the end of the title to be more specific.
Also, in the ``Competition'' section you mention competition with vertebrates and micro-organisms, but not with invertebrates.
You should consider them in this section (and in ``Facilitation''; see next point).}

\textcolor{blue}{We changed the title of the manuscript to ``A Recipe for Scavenging in Vertebrates - the natural history of a behaviour''. We have added two mentions of inverebrates as competitors to the manuscript but given our focus these are not expanded upon.}
%TG: Also maybe mention invertebrates in the competition part of the manuscript. + For a response to the reviewer, here is probably a good point for stating clearly that we wanted this manuscript to be focusing more on vertebrates because of the palaeo aspect. Both micro-organisms and invertebrates (I guess mainly arthropods) are unlikely to display preserved morphological clues for suggesting a scavenging behaviour because of their poor preservation anyway.

%KHwhat line are the mentions of invertebrates on. Do insects help facilitate through breaking open carcasses or anything?


\item{I found difficult to follow the main point of the paper because the ms is not well organized. Following a hierarchical order, I'd define four/five main sections: ``Introduction'', ``The Challenges of Scavenging'' (this could be a subsection within the previous section), ``Encounter Rate'' (with four different subsections: ``Metabolism'', ``Locomotion'', ``Sensory Detection'' and ``Carcass Availability''), ``Handling Time'' (which would include ``Food Processing'') and ``Conclusion''. \\ 

Moreover, ``Competition'' and ``Facilitation'' (I strongly encourage you to consider not only competitive, but also facilitative processes, which are mostly neglected in your review) should be treated as transversal factors that can modulate each of the abovementioned parameters. Thus, I'd include them in a different section, or mention the competitive and facilitative processes related to each parameter within each subsection. Fig. 1 should be re-organized accordingly.}

\textcolor{blue}{We clarified the structure of the manuscript by introducing clearer hierarchical subsections, an additional section on facilitation and a revised figure 1.}
%TG: + more blabalal on his ``Facilitation'' part

%%KH I think a bit more needs to be added to this section. Something as simple "While such facilitation is difficult to detect in extinct species the presence of facilitatory species may indicate the possibility of the interaction. For example, early Hominds may have benefited from the presence of vultures, and in fact many  scavenging species now depend on human refuse as a means of scavenging in current systems. Any other examples?"

\item{The use of the scientific literature on scavenging is deficient, especially for a review paper like this. A number of key references are missing, e.g., Selva \& Fortuna (2007 Proc. R Soc. B 274:1101-1108), Wilson \& Wolkovich (2011 TREE 23:129-135), Beasley et al. (2012 Oikos 121:1021-1026), Cort\'{e}s-Avizanda et al. (2012 Ecology 93:2570-2579), Cort\'{e}s-Avizanda et al. (2014 Ecology 95:1799-1808), Mole\'{o}n et al. (2014 Biol. Rev. 89:1042-1054), Pereira et al. (2014 Mammal Rev. 44:44-55), P\'{e}riquet et al. (2015 Biol. Rev. 90:1197-1214), Mateo-Tom\'{a}s et al. (2015 Divers. Distrib. 21:913-924), Mole\'{o}n \& S\'{a}nchez-Zapata (2015 BioScience 65:1003-1010), Moreno-Opo et al. (2016 Behav. Ecol.), Sebasti\'{a}n-Gonz\'{a}lez et al. (2016 Ecology 97:95-105). Your literature synthesis might benefit from the reading of these articles, for instance regarding competitive and facilitative interactions among scavengers. Also, the review would be more informative and useful for readers. Next I mention some particular comments regarding the use of the references (I mention others in Minor comments):}\\

\textcolor{blue}{Thanks very much for these very useful references, all of which are now included and used to bolster some sections of the paper, particularly with respect to facilitation, see below for further details.}

Page 2, 1st sentence: but see Mole\'{o}n \& S\'{a}nchez-Zapata (2015); many of the most charismatic vertebrates of the world are scavengers. Rather than ``scavengers'', what is not charismatic is probably ``scavenging''.

\textcolor{blue}{Our introduction has been rewritten and this passage no longer appears.}

 Page 2, 13th line: Mole\'{o}n \& S\'{a}nchez-Zapata (2015) may be more appropriate than Koenig (2006) because the former is more general and not restricted to vultures.\\
 Page 2, lines 17th-18th: see Pereira et al. (2014) and P\'{e}riquet et al. (2015) for reviews.

\textcolor{blue}{These references have been added.}

 Page 3, bottom paragraph, bottom line: this is not true, as scavenging may indirectly affect herbivore populations and thus carrion availability (see Mole\'{o}n et al. 2014 Biol. Rev.).

\textcolor{blue}{The potential for scavengers to affect carrion availability has been acknowledged.}

 Page 5, bottom paragraph, 1st sentence: Mateo-Tom\'{a}s et al. (2015) is more general than Kendall (2013).\\
 Page 8, 1st and 2nd sentences: see Pereira et al. (2014) and P\'{e}riquet et al. (2015).\\
 Page 8, 3rd sentence: you may also cite DeVault et al. (2003) and Pereira et al. (2014).

\textcolor{blue}{These references have been added.}

 Page 8, 8th-17th lines: please reword; wild dogs hardly scavenge, while leopards and lions are frequent scavengers (see Pereira et al. 2014).

\textcolor{blue}{We have reworded this section to make the more general point that ambush predators can rely more on hunting than can cursorial species, a point explicitly stated in the Pereira et al. (2014) article, which we now cite.}

 Page 11, 1st sentence: see Kane et al. (2014) and Mole\'{o}n et al. (2014 Biol. Rev.) for inter-specific interactions.\\
 Page 11, 4th sentence: Kane et al. (2014) is not appropriate here because it is about inter-specific interactions; use Cort\'{e}s-Avizanda et al. (2014) instead.\\
 Page 14, 2nd paragraph, 1st sentence: see also Mole\'{o}n et al. (2014 Biol. Rev.) and Pereira et al. (2014).\\
 Page 14, 2nd paragraph, 2nd sentence: see better Pereira et al. (2014) and P\'{e}riquet et al. (2015). \\
 Page 14, bottom paragraph, 3rd sentence: you may also consider Moreno-Opo et al. (2016).\\
 Page 15, 2nd paragraph, 2nd sentence: see also Mole\'{o}n et al. (2014 Biol. Rev.) and Pereira et al. (2014).

\textcolor{blue}{All of these references have been added.}

\item{You should include more discussion about the importance of carcass size throughout the ms, as it has a strong influence on scavenging patterns and interactions among vertebrates and between vertebrates and smaller carrion consumers (e.g., see Mole\'{o}n et al. 2015).}

\textcolor{blue}{We have included a new section detailing the influence of carcass size on scavenger ecology \textit{``Modern scavenging assemblages are known to be influenced by carcass size (Moleon et al. 2015). 
Larger carcasses tend to last longer and also present a more conspicuous target for a foraging scavenger which results in more species attending them (Moleon et al. 2015).
This will have had implications for extinct assemblages because body size distributions vary across different environments but also across time. 
O'Gorman \& Hone (2012) showed Mesozoic faunal distributions may have been skewed towards larger species even when fossil biases are taken into account. 
Similarly, the megafauna of the Pleistocene (Doughty et al. 2013) would have produced large carcasses. 
As a result, the scavenger assemblages during this era would have been particularly diverse.''}}

\end{enumerate}

\subsection{Minor comments:}

\begin{enumerate}

\item{GENERAL:} Please provide line numbers to facilitate the reviewing process.

\textcolor{blue}{Line numbers have now been included. }

\item{ABSTRACT:} You could remove ``, the first to our knowledge''.

\textcolor{blue}{This has been removed.}

\item{ABSTRACT:} Also, you could mention some relevant specific findings of your synthesis.

\textcolor{blue}{We cannot provide details of specific predictions in sufficient detail to be intelligible within the work limit of the abstract, but have modified the structure of the manuscript to make these stand out more clearly.} 


\item{THE CHALLENGES OF SCAVENGING:} Page 3, 1st sentence: ``often difficult to predict'' compared to what? Please specify. There are many examples in which carrion is highly predictable, e.g., during salmon spawning (see also Pereira et al. 2014).

\textcolor{blue}{This passage has been removed from the section and later on in the paper we note that: \textit{``Carrion can often be unpredictable, such as in cases where animals succumb to disease''}}

\item{ENCOUNTER RATE:} Page 3, bottom paragraph, 3rd line: change ``Alternatively'' to ``Also''?

\textcolor{blue}{This has been changed.}

\item{METABOLISM:} Page 4, 1st sentence: ``the sporadic nature of carrion''; please explain.

\textcolor{blue}{This has been changed to ephemeral.}

\item{LOCOMOTION:} Page 4, bottom paragraph, 1st sentence: change ``inherent'' to ``relative''.

\textcolor{blue}{This has been changed.}

\item{LOCOMOTION:} Page 4, bottom paragraph, 3rd sentence: remove ``, paradoxically,''.

\textcolor{blue}{This has been deleted.}

\item{LOCOMOTION:} Page 5, 2nd paragraph, 1st sentence: change ``appear'' to ``appeared''. Also, specify Mya for the Cretaceous period.

\textcolor{blue}{This has been changed.}

\item{LOCOMOTION:} Page 9, top paragraph: please be more cautious in your assertion, as cooperation skills and tools of early hominins probably allowed them to defend and usurp carcasses from more powerful scavengers.

\textcolor{blue}{We have changed this paragraph and take into account the authors' reference to tool use and sociality in hominins. The point is that even with these advantages, running to a carcass over long distances still wasn't energetically sensible.}

\item{DETECTION:} Page 11, top and 2nd paragraph, 1st sentence: this is about facilitation. As said above, facilitative processes may affect other parameters as well, not only the ability to detect carrion.

\textcolor{blue}{Facilitation is now considered throughout the manuscript.}

\item{DETECTION:} Page 11, last sentence: explain why wind speed affects the olfactory capacity of an aquatic species.

\textcolor{blue}{We have added a section explaining why this is so: \textit{``wind speed determined the number of sharks feeding at whale carcasses due to chemical stimului from the carcasses being propogated through the water by the wind, indicating they were dependent on detecting the odours from the decaying whales.''}}

\item{PREY AVAILABILITY:} I'd change this title to ``Carcass Availability'', which is more specific.

\textcolor{blue}{This has been changed.}

\item{PREY AVAILABILITY:} Among the different components of carcass availability, e.g., abundance, duration, predictability and accessibility, in this section you only talk about abundance and duration. You should also discuss about carcass predictability (e.g., see Cort\'{e}s-Avizanda et al. 2012) and carcass accessibility (e.g., see DeVault \& Krochmal 2002).

\textcolor{blue}{We now discuss all 4 aspects e.g. \textit{``Notably, freezing carcasses can become too hard to consume by most vertebrate carnviores (Selva et al 2003)''} and \textit{``Spikes in temperature can result in severe droughts causing mass mortality events which result in relatively predictable peaks in carrion availability (Kendall et al. 2014).''}}


\item{PREY AVAILABILITY:} Page 12, 2nd paragraph, 1st sentence: as said before, this is not true, given that scavenging may indirectly affect herbivore populations and thus carrion availability (see Mole\'{o}n et al. 2014 Biol. Rev.).

\textcolor{blue}{We have removed the part of the sentence where we mistakenly suggest that scavengers cannot affect carrion availability.}

\item{PREY AVAILABILITY:} Page 13, 3rd and 4th sentences: whaling has exerted also a huge negative impact on scavengers by depleting whale populations. If you are going to talk about human impacts, maybe is better to do it at the end of the ms, perhaps in a new section in which you explore which scavenging traits could be more successful in a human dominated world.

\textcolor{blue}{We have decided to remove this section because we felt human impacts go beyond the scope of our paper.}

\item{PREY AVAILABILITY:} You should also take into account facilitative processes. For instance, some scavengers are able to access the interior of carcasses only once other, more powerful species, has opened the skin (see Mole\'{o}n et al. 2014 Biol. Rev.).

\textcolor{blue}{Our paper now references the literature on the facilitative processes involved in scavenging.}

\item{PREY AVAILABILITY:} Page 14, bottom paragraph, 2nd sentence: I'd change it to ``This is true for many vultures and other major avian scavengers such as albatrosses who all have […]''.

\textcolor{blue}{We have reworded the sentence as suggested.}

\item{FOOD PROCESSING:} Page 16, 2nd paragraph, 6th sentence: insert ``Within mammals,'' before ``This ability''.

\textcolor{blue}{This has been added.}

\item{FOOD PROCESSING:} Page 17, 2nd paragraph, 4th sentence: do you refer to species or individuals? Higher species diversity doesn't mean more individuals.

\textcolor{blue}{The reference of Pobiner (2015) states that there were more individuals and we have changed the sentence to reflect this.}

\item{FOOD PROCESSING:} Page 18, 1st sentence: change ``enabling'' to ``enable''.

\textcolor{blue}{This has been changed.}

\item{FOOD PROCESSING:} Page 18, 3rd sentence: ``thus''? I don't understand this sentence; could you please reword it?

\textcolor{blue}{We have reworded this sentence. The idea is that early birds didn't have beaks and so could more easily eat meat. This is in contrast to modern birds where a beak may be a hindrance.}

\item{FOOD PROCESSING:} Page 18, 5th sentence: ``The mix of strong and weak features in their skull morphology''; please explain it.

\textcolor{blue}{We now note that their \textit{``skull morphologies and neck musculature are indicative of animals that were suited to removing large amounts of flesh from an immobile foodstuff (Witton 2013)''.}}

\item{CONCLUSION:} Page 18, bottom paragraph, 3rd sentence: change ``technique'' to ``approach''?

\textcolor{blue}{This has been changed.}

\item{FIGURES:} Fig. 2: you may insert the name and Mya range of each period in the X axis rather than in the figure legend. Also, I'd indicate ``aerial'', ``terrestrial'' and ``aquatic'' in the Y axis rather than in the legend.

\textcolor{blue}{This figure has been removed}

\end{enumerate}

Hope this helps,
Marcos Mole\'{o}n

\bigskip
\bigskip
\bigskip
\bigskip
\bigskip
\bigskip
\bigskip
\bigskip

Yours Sincerely,
\bigskip

Adam Kane and co-authors 

%\closing{Yours Faithfully,}
%\ps{P.S. Here goes your ps.}
%\encl{Enclosures.}
%\end{letter}
\end{document}